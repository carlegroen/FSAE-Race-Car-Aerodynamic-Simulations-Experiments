% !TEX root = main.tex
\chapterstyle{abstract}
\vspace{6cm}
\chapter*{Abstract}

The ability to aerodynamically improve grip without adding a weight penalty is key to winning with race cars. High downforce is highly sought after, and the newly started Vermilion Racing Team at DTU is no exception. This work presents the theoretical arguments showing why downforce is important and numerical simulations to optimize the proposed rear wing dimensions. The numerical simulations are held up against a wind tunnel experiment, showing how misalignment of wings can greatly interfere with airflows, confirming the complexity of designing aerodynamical parts. The result is a design specifications of an easily producible rear wing with two elements and large end plates providing an improvement in cornering speeds between $3-35\%$, depending on the turn radius.

\chapterstyle{box}
