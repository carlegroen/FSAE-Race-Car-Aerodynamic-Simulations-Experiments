% !TEX root = main.tex
\chapter{Introduction}

\emph{Vermilion Racing} is a newly started Electric race car team building their first vehicle: The Eevee \cite{bulba}. The teams' purpose is competing against other Universities at the Silverstone race track from the 11\textsuperscript{th} to the 16\textsuperscript{th}. As members of the team, the purpose of this report is to document the design process of the rear wing of the first car, the Eevee, and provide an aerodynamic package documenting drag and downforce. The intent is to start a student organization, passing on the teachings of racecar mechanics for many years to come.

Aerodynamics is a major decider in racing today. Cornering, not top speed is the deciding factor amongst the teams, and aerodynamics is the key. Drag, lift and side force are the three cornerstones to vehicle aerodynamics. A car's ability to handle depends on the grip of the tyres, and downforce directly increases grip by increasing the downwards load on the tyres without adding a weight penalty. Additionally, drag directly decreases the speed of a vehicle by increasing air resistance, but is of less importance as the car's in this class have far more accelerative power than the tyres can handle \cite{jkatz}. Designing the bodyworks of Eevee is therefore a dance of downforce.

The following concerns the creation and optimization of the Eevee's rear wing. The first chapter concerns a comprehenseive walkthrough of the theory that restricts the design of the car's aerodynamics. The second chapter establishes the necessary experimental ground work to verify the numerical simulations performed later on -- a wind tunnel test of a 1/4 scale model of the wing profile. The data received from the experiment gives a pressure profile along the wing profile. The same wing is then simulated in Star-CCM+, followed by a comparison of the two results to verify the correctness of the simulations. The size, $x$- and $y$-distance between the multi-element wings, angle of attack and height relative to the chassis is then optimized for maximum downforce. The optimized wing is finally produced with strength calculations of the maximum deflection to stay in accordance to the rules of the competition. Finally, the end result is discussed and possible improvements to the wing and mounting system is listed. 



\section{Motivation}
- Why are we designing this to begin with

\section{Design Philosophy}


Designing a car with hundreds, if not thousands different factors is incredibly difficult. Therefore, analyzing which things matters and which don't is crucial to the teams' success. First, the deadline for finishing the wing is soon. This sets a limit on the complexity of the wing's dimensions. Secondly, this is the first iteration of the car, thus no decision can be based on prior experience, and optimization has to wait for the next iteration. Finally, the finances of the car is tight. This forces us to go with a low-cost solution. We therefore have to go for a simple, low cost wing, basing the design solely on litterature and experience other teams have done.

The wing has to have as low weight as possible, in order to ensure optimum acceleration of the car. Thereto, drag has to be kept low, but the effect from drag is near negligible, which will be explained in section \ref{sec:topspeed}. Lastly, we have to maximize the downforce provided by the wing, but also ensure that the center of pressure is kept as constant as possible. If the frontwing, undertray and rear wing's downforce don't scale equally with speed, the center of pressure will move during acceleration. This will make the car's handling unpredictable, and potentially limit the driver's confidence in the car.

\section{Design restrictions} %Maybe move to another spot
