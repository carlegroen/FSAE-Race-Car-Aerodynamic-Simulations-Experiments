% !TEX root = main.tex
\chapter{Introduction}

\emph{Vermilion Racing} is a newly started Electric race car team building their first vehicle: The Eevee \cite{bulba}. The teams' purpose is competing against other Universities at the Silverstone race track from the 11\textsuperscript{th} to the 16\textsuperscript{th}. As members of the team, the purpose of this report is to document the design process of the rear wing of the first car, the Eevee, and provide an aerodynamic package documenting drag and downforce. The intent is to start a student organization, passing on the teachings of racecar mechanics for many years to come.

Aerodynamics is a major decider in racing today. Cornering, not top speed is the deciding factor amongst the teams, and aerodynamics is the key. Drag, lift and side force are the three cornerstones to vehicle aerodynamics. A car's ability to handle depends on the grip of the tyres, and downforce directly increases grip by increasing the downwards load on the tyres without adding a weight penalty. Additionally, drag directly decreases the speed of a vehicle by increasing air resistance, but is of less importance as the car's in this class have far more accelerative power than the tyres can handle \cite{jkatz}. Designing the bodyworks of Eevee is therefore a dance of downforce .



\section{Motivation}
- Why are we designing this to begin with

\section{Design Philosophy}
- what are we designing for? low weight, high downforce. drag a bit neglegible due to high power motors.

\section{Design restrictions} %Maybe move to another spot
