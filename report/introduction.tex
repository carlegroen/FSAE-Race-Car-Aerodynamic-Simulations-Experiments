% !TEX root = main.tex
\chapter{Introduction}
  \emph{Vermilion Racing} is a newly started Electric race car team building their first vehicle: The Eevee \cite{bulba}. The teams' purpose is competing against other Universities at the Silverstone race track from the 11\textsuperscript{th} to the 16\textsuperscript{th}. As members of the team, the purpose of this report is to document the design process of the rear wing of the first car, the Eevee, and provide an aerodynamic package documenting drag and downforce.

  Aerodynamics is a major decider in racing today. Cornering, not top speed is the deciding factor amongst the teams, and aerodynamics is the key. Drag, lift and side force are the three cornerstones to vehicle aerodynamics. A car's ability to handle depends on the grip of the tyres, and downforce directly increases grip by increasing the downwards load on the tyres without adding a weight penalty. Additionally, drag directly decreases the speed of a vehicle by increasing air resistance, but is of less importance as the car's in this class have far more accelerative power than the tyres can handle \cite{jkatz}. Designing the bodyworks of Eevee is therefore a dance of downforce.

  This bachelor thesis attempts to lay down the ground work for designing, optimizing and manufacturing the car's rear wing. Our advisor, Jens Honore Walther, proposed simulating the wing geometry in order to both cheapen and quicken the optimization process, but also gain insight into how flow in front of the rear wing affects lift performance.  The purpose of the numerical simulations are ultimately to create an easily producable wing in a very short timespan, that produces ample amounts of negative lift for the time spent in production. \fxnote{write about what has been done prior}

  As this is the first year the team building a car, time and money for production is sparse. To ensure the simulated optimizations are true, a wind tunnel test is performed on a down-scaled wing using Flow Similarity theory. The measurements are compared with the results from a computational fluid dynamics simulation performed in Star-CCM+ to verify the preciseness of the simulations. Based on this, the size, $x$- and $y$-distance between the multi-element wings, angle of attack and height relative to the chassis is then optimized for maximum downforce. The multi-element wing is taken from theoretical abstraction to reality with a physical design of the wing. The position, deflection and dimensions of the wing is thoroughly described by the rules of the Formula Student competition, guiding the final design process. In the final chapter, theories regarding strength of sandwich structured composites and carbon fiber molding are handled, in order to descibe the production choices made. Finally, the end result is discussed and possible improvements to the wing and mounting system is listed.

  Measurements and tests have been carried out at the DTU Wind laboratory's Red wind tunnel with the help of our supervisor Robert Flemming Mikkelsen the 19\textsuperscript{th} of June.

  The entire project is publicly available on \url{https://github.com/carlegroen/bachelor_project_racecar_aero}, where all work files, data, various notes and CAD drawings can be found. The Formula Student team at DTU: Vermilion Racing can be followed on \url{fb.com/FSDTU}.

\section{Design Philosophy}
  Designing a car with hundreds, if not thousands different factors is incredibly difficult. Therefore, analyzing which things matters and which don't is crucial to the teams' success. First, the deadline for finishing the wing is soon. This sets a limit on the complexity of the wing's dimensions. Secondly, this is the first iteration of the car, thus no decision can be based on prior experience, and optimization has to wait for the next iteration. Finally, the finances of the car is tight. This forces us to go with a low-cost solution. We therefore have to go for a simple, low cost wing, basing the design solely on litterature and experience other teams have done.

  The wing has to have as low weight as possible, in order to ensure optimum acceleration of the car. Thereto, drag has to be kept low, but the effect from drag is near negligible, which will be explained in section \ref{sec:topspeed}. Lastly, we have to maximize the downforce provided by the wing, but also ensure that the center of pressure is kept as constant as possible. If the frontwing, undertray and rear wing's downforce don't scale equally with speed, the center of pressure will move during acceleration. This will make the car's handling unpredictable, and potentially limit the driver's confidence in the car.
