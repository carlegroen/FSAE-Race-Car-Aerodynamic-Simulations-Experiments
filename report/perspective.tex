% !TEX root = main.tex
\chapter{Perspective}
\label{chap:perspective}

The Eevee has to be rebuilt next year, and in order to help the effort along for future students, a list of potential upgrades are listed below, as well as a quick overview of how to improve pre-production design.

\section{Simulations}

For optimizing the design of the rear wing, using software such as CAESES allows for very quick optimization of parameters - something that would be essential to bring into the routine. Thereto, creating a full aerodynamic package with both body panels, driver, front wing and rear wing is essentialf or optimizing the entire car. Ground effects and the open wheel structure can also greatly change the lifting abilities of the car, which is why simulating the entire vehicle is important.

\section{Drag Reductive System}

A drag reductive system (DRS) is well-known from Formula 1, and has in the recent years been gaining traction (or lack therof) in the Formula Student. It is a natural extension to the aerodynamics of the car, as Formula Student has much less restrictions on aerodynamics than Formula 1. Automatically adaptive DRS, that measures the car's relative downforce and the angle of the steering column could give a big edge on straights, as flipping the wing up to reduce drag increases the top speed.

\section{Slats, Flaps, Gills and Cutaway}

%https://www.jmranalytical.com/single-post/2017/04/06/Rear-Wing-Investigation

We already have a multi element wing, but increasing the amount of elements increases the amount of downforce we can pull out of the same design. Thereto, adding slats, will allow for even higher angles of attack, albeit not increasing the $C_L$ further.

Cutting away some of the wing's end plate can also introduce beneficial vortices, or alter the way air flows over the plates. 

\section{Suspension Integration}

For optimum grip, having the rear wing press down directly on the tyres would be a great addition. Implementing a mount directly in the upright is the optimum solution, and with a low enough wing weight, adding extra unsprung mass is not a big issue.

\section{Load Distribution}

As mentioned in chapter \ref{chap:vehicleperformance}, load distribution is essential to the car's handling. If the lift coefficient differs too much between the front- and rear wing, the car's center of pressure will move with speed. This is very unattractive for inexperienced drivers, for which the competition require us to be.

\section{Pre-preg Carbon Fiber}

While the hand layup went well, moving on to pre-preg carbon fiber mattes will make for an even stronger wing with less weight. Thereto, fiber direction is guaranteed to be optimum, making for a safer product.
