% !TEX root = main.tex
\chapter*{Perspective}

The Eevee has to be rebuilt next year, and in order to help the effort along for future students, a list of potential upgrades are listed below, with estimates of how valuable each change is in regards to downforce/drag reduction gain.

\section{Drag Reductive System}

A drag reductive system (DRS) is well-known from Formula 1, and has in the recent years been gaining traction (or lack therof :))) ) in the Formula Student. It is a natural extension to the aerodynamics of the car, as Formula Student has much less restrictions on aerodynamics than Formula 1. Automatically adaptive DRS, that measures the car's relative downforce and the angle of the steering column could give a big edge on straights, as flipping the wing up to reduce drag increases the top speed.

\section{Slats, Flaps, Gills and Cutaway}

https://www.jmranalytical.com/single-post/2017/04/06/Rear-Wing-Investigation

We already have a multi element wing, but increasing the amount of elements increases the amount of downforce we can pull out of the same design.

\section{Suspension Integration}

- Nice to have downforce directly on the wheels
- Gives more unsprung mass though. That might be an issue.
