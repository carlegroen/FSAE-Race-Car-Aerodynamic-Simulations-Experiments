% !TEX root = main.tex
\chapter{Airfoils and Inverted Wings}

Achieving a large negative lift coefficient $C_L$ can be done in many ways. Inspecting race cars throughout the years show that airfoils have been used as early as 1966 when Jim Hall attached a rear wing to his Chaparral 2E \cite{hucho}. Since then, the inverted wings have been a staple in the racing industry with various three dimensional geometries affecting the overall performance even further.

This chapter covers the pressure distribution of various airfoils, the selection criterions of the competition, three dimensional geometrical effects and the tools of the optimization trade.

\section{Airfoil theory}

  An airfoils is the 2-dimensional cross section of a wing, that's characteristic of the wing's lifting characteristics. It is important to know the nomenclature: The leading edge is  the most forward point of the wing, the trailing edge is the most rearward point of the wing. Camber is how much the wing ''flexes''.




  \fxnote{Mostly laminar flow, boundary layer mustn't trip or create bubbles,}
Theory of airfoils from katz book, to be written tuesday 19.



  \subsection{Pressure distribution}


  \fxnote{Where does downforce come from? Where does drag come from?}

  \subsection{Lift and Drag Coefficient}

  \subsection{Which parameters does an airfoil have? What can we change}

  \subsection{Angle of Attack}

  \subsection{Ground Effects}\fxnote{unsure}

  \subsection{Aspect Ratio and End Plates}

    An important identifyer when describing the entire wing is, apart from chord length and airfoil design, the width. The definition used for describing the physical span is called Aspect Ratio, and for a rectangular wing is:
    \begin{align}
        \AR_\text{actual} &= \frac{b}{c}
        \label{eq:ARactual}
    \end{align}
    The end plates adds a virtual additional length, by reducing the vortices that usually go around the wing and reduce lift. A corrected aspect ratio can be found for wings with side plates as:
    \begin{align}
      \AR &= \AR_\text{actual}\left(1+1.9\frac{h}{b}\right)
    \end{align}
    where $h$ is the height of the end plate, and $b$ is the width of the wing as in equation \ref{eq:ARactual} \cite{jkatz}.

  \fxnote{something about Navier Stokes equations.}

\section{Comparison of Airfoils}
Choosing the MSHD wing.
To be written from MSHD article on tuesday 19.

\section{Multi elements? How many is enough}
From katz book, written tuesday 19.

\section{Optimization Tools}

Also from katz book to be written tuesday 19.

Building physical models is cool, wind tunnel time is expensive though

Let's use CFD!

To make sure CFD works, we need to perform experiments to verify simulations work.

Let's make a small scale wind tunnel test and simulate the rest!
