% !TEX root = main.tex
\chapter{Discussion}
  The following section will discuss the results in chapter \ref{sec:experiments} and the simulations made in chapter \ref{sec:simulations}. The purpose of this paper was to investigate the effects or aerodynamics on race cars, design a solution that would improve the car's lap time and finally produce the hypothesized aerodynamic device.

  \section{Theory, Experiments and Simulations}

  The wind tunnel results plotted in figure \ref{fig:clperAOAexperiment} shows a correlation between the wing's lift and theoretical lift, assuming a constant $C_L$ for the theoretical wing. The experimental results are all lower than the theoretical, which may be due to several facts: First, the wing section's alignment at higher speeds was interrupted, as one section was skewed $\SIrange{2}{3}{\milli\metre}$, creating a small gap between the wing sections. Secondly, the wing's extremely high lift interferes with the flow behaviour, as the wing's wake is most likely pushing air against the walls. This theory is corroborated by the simulations as seen in figure \ref{fig:scalewingwindtunnelsim}, where it is clearly seen that the wake interferes with the wall's boundary layer. Third, and most likely the most important, the placement relative to each other may be slightly off. Creating a small multi element airfoil is very difficult, as placement is everything to the lift of the wing. According to the wing optimization process seen in figure \ref{fig:multieleoptimization}, very small changes can rapidly change the  lifting characteristics of the wing. This could explain why the lift of the airfoil model is much lower than the theoretical, and the fact that the simulations coincide very well with theory. The lesson learned is that the first element must be very carefully placed relative to the second element, in order to not alter the flow substantially.




\section{Product Design Specification Review}

  The finished product have to live up to the design specification, in order to be a useful solution. In table \ref{tab:designreview}, it can be seen that all reqirements are fulfilled, and all criteria except one. During the design process of the car, the removal of the engine became negligible as the battery pack is going to be removed in another way. This voids the criteria, and thus makes the solution an optimal one.

  \begin{table}
    \begin{tabularx}{\textwidth}[t]{>{\columncolor{seapurple!40}}l XX}
      \arrayrulecolor{seapurple}\hline
      \rowcolor{white}
      \textbf{\textcolor{seapurple}{Issue}} & \textbf{\textcolor{seapurple}{Requirement}} & \textbf{\textcolor{seapurple}{Criteria}}\\
      \hline
      Weight & \cellcolor{seagreen!40}Must not move CM above halfway point & \cellcolor{seagreen!40}Should be as low as possible \\
      Safety & \cellcolor{seagreen!40}Must be in compliance with FSAE rules & \cellcolor{seagreen!40}Should not make handling difficult for driver\\
      Durability & \cellcolor{seagreen!40} Must have no fatigue limit. Must to be waterproof \\
      Performance & \cellcolor{seagreen!40} High downforce \& soft stall characteristic at all speeds &\cellcolor{seagreen!40} Should retain perfomance despite tripping. Should have end plates.\\
      Dimensioning & \cellcolor{seagreen!40} Must be within area defined by FSAE rules & \cellcolor{seayellow!40} Should allow space for motor removal. \\
      Production & \cellcolor{seagreen!40} Low time- and monetary cost
      \label{tab:designreview}
    \end{tabularx}
    \caption{The PDS table shows how the final design lives up to the proposed specifications.}
  \end{table}

  During the productio
