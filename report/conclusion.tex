% !TEX root = main.tex
\chapter{Conclusion}
  Vermilion Racing's rear wing has been designed based on initial research, wind tunnel experiments and simulations. Each step has been fundamental to the understanding of the importance of the theory, and will serve as a tool for future generations to expand upon.

  The rear wing's dimensions are restricted by the rules of the competition, and serve as the boundaries of the design, giving us the width and height of the wing. The wing's airfoils are chosen based on XFOIL simulations of lift profiles. Due to the nature of the competition, a highly stable wing is wanted - stalling is fatal to the car's handling abilities, which is why the highly cambered MSHD airfoil was chosen. Increasing the amount of elements greatly increases the possible angles of attack, and based on theory and time constraints, a two-element wing with identical airfoils serves as the best opportunity for increasing the lift.

  The initial concept was tested in a wind tunnel on a quarter scale model, where the results were compared to simulations. An irregularity in the results instigated further examination of the scale model, revealing a misplacement of the second wing element. After simulating their relative placement, it was discovered that relative placement is very important for lift. Therefore, new wind tunnel tests should be performed in order to verify the lifting characteristics of the wing.

  After simulating and optimizing position, a proposed design was drawn. Metallic inserts in order to reinforce the wing and physically mount it to the end plates were topologically optimized to reduce weight. The final design of the rear wing, combined with initial simulations of the remainder of the car shows to give a total down force of $C_L = 2.4$, which should give an increase in cornering speed of $3-35\%$, based on the cornering radius.

  Finally, a series of propositions for improving the rear wing design can be read in chapter \ref{chap:perspective}. Many of these should bring great improvements to the Eevee's lift characteristics.
