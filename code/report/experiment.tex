% !TEX root = main.tex
\chapter{Experiment}




Verifying the simulated aerodynamic effects is crucial to ensuring the correctness of the numerical analysis. In order to assess the reliability of the previouisly conducted simulations, a physical measurement of the pressure along the down-scaled wing.

In figure \ref{fig:scalewing} the down-scaled wing can be seen with pressure-taps along the centerpiece.



\section{Equipment}

\begin{itemize}
  \item The Red wind tunnel ($\SIrange{60}{65}{\metre\per\second}$)
  \item 1/4 scale wing
  \item Syringe inserts
  \item Rubber tubing
  \item Pressure transducer
\end{itemize}

\subsection{1/4 Scale Wing}

The small scale rear wing is constructed in 6 pieces. The large wing is dissected into three parts. Two regular wings, and a central part with 15 pressure taps.

Material selection is based on the ease of machinability - a CNC-miller was provided to us, along with ample amounts of aluminium. This scale wing is not to be used in the actual race car, so weight is not a concern. Construction the 1/4 scale wing is not completely trivial. High precision is required for the surface finish, and the pressure taps have to be small in diameter: $Ø\SI{0.8}{\milli\metre}$. \fxnote{insert work drawings used for producing the wing}.

\section{Experimental Procedure}

Given the nature of the competition, the average cornering speeds are around $\SI{55}{\kilo \meter \per \hour} = \SI{15.28}{\metre\per\second}$, which is where downforce is of most importance. As shown in section \ref{sec:similarflows}, the desired velocity in the wind tunnel for the scale model can be found from equation \ref{eq:windtunnelspeed}

\begin{align*}
  u_\text{m} &= \frac{\SI{0.6}{\metre}}{\SI{0.15}{\metre}} \SI{15.28}{\metre\per\second} = \SI{61.12}{\metre\per\second}
\end{align*}

Which in accordance to the range of The Red wind tunnel.


\section{Results}
